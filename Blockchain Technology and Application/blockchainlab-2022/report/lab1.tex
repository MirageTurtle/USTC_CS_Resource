\documentclass{article}
\textheight 23.5cm \textwidth 15.8cm
%\leftskip -1cm
\topmargin -1.5cm \oddsidemargin 0.3cm \evensidemargin -0.3cm
%\documentclass[final]{siamltex}

\usepackage{ctex}
\usepackage{graphicx}
\usepackage{epsfig}
\usepackage{amsmath}
\usepackage{amsthm}
\usepackage{amssymb}
\usepackage{mathrsfs}
\usepackage{float}
\usepackage{multirow}
\usepackage{verbatim}
\usepackage{fancyhdr}
\usepackage{subfigure}
\usepackage{color}
\usepackage{mathtools}
\usepackage{mathrsfs}
%\usepackage{natbib}
\usepackage{sectsty}
%\usepackage[title]{appendix}
\usepackage{threeparttable}
\usepackage{dcolumn}
\usepackage{booktabs}
\usepackage{indentfirst}
\usepackage{setspace}
\usepackage{bm}
\usepackage{enumerate}
\usepackage{geometry}
\usepackage{listings}
\usepackage{fontspec}
\newfontfamily\jetbrains{JetBrains Mono}
\lstset{
    breaklines,                                 % 自动将长的代码行换行排版
    extendedchars=false,                        % 解决代码跨页时,章节标题,页眉等汉字不显示的问题
    backgroundcolor=\color[rgb]{0.96,0.96,0.96},% 背景颜色
    keywordstyle=\color{blue}\bfseries,         % 关键字颜色
    identifierstyle=\color{black},              % 普通标识符颜色
    commentstyle=\color[rgb]{0,0.6,0},          % 注释颜色
    stringstyle=\color[rgb]{0.58,0,0.82},       % 字符串颜色
    showstringspaces=false,                     % 不显示字符串内的空格
    numbers=left,                               % 显示行号
    numberstyle=\tiny\jetbrains,                    % 设置数字字体
    basicstyle=\small\jetbrains,                    % 设置基本字体
    captionpos=t,                               % title在上方(在bottom即为b)
    frame=single,                               % 设置代码框形式
    rulecolor=\color[rgb]{0.8,0.8,0.8},         % 设置代码框颜色
}
\usepackage{hyperref}
\hypersetup{
    hypertex=true,
    colorlinks=true,
    linkcolor=black,
    anchorcolor=black,
    citecolor=black,
}

\title{区块链实验一 实验报告}
\author{PB19071405 王昊元}
\date{2022 年 04 月 10 日}

\begin{document}
\maketitle

\section{实验目的}

\begin{enumerate}
    \item 完成\emph{SHA256}算法实现
    \item 实现\emph{Merkle树}的构建
    \item 搭建简单的区块链结构
\end{enumerate}

\section{具体实现}

\begin{enumerate}
    \item \emph{SHA256}算法 \\
    算法中涉及到几种运算符,如\jetbrains{rightRotate}、\jetbrains{Sigma0}、\jetbrains{ch}、\jetbrains{maj}等,
    通过新定义的函数实现,下面仅展示核心函数\jetbrains{mySha256}相关代码,该函数按照官方算法实现。
    \begin{lstlisting}[language=go]
func mySha256(message []byte) [32]byte {
    //前八个素数平方根的小数部分的前面32位
    h0 := uint32(0x6a09e667)
    h1 := uint32(0xbb67ae85)
    h2 := uint32(0x3c6ef372)
    h3 := uint32(0xa54ff53a)
    h4 := uint32(0x510e527f)
    h5 := uint32(0x9b05688c)
    h6 := uint32(0x1f83d9ab)
    h7 := uint32(0x5be0cd19)

    //自然数中前面64个素数的立方根的小数部分的前32位
    k := [64]uint32{
        0x428a2f98, 0x71374491, 0xb5c0fbcf, 0xe9b5dba5, 0x3956c25b, 0x59f111f1, 0x923f82a4, 0xab1c5ed5,
        0xd807aa98, 0x12835b01, 0x243185be, 0x550c7dc3, 0x72be5d74, 0x80deb1fe, 0x9bdc06a7, 0xc19bf174,
        0xe49b69c1, 0xefbe4786, 0x0fc19dc6, 0x240ca1cc, 0x2de92c6f, 0x4a7484aa, 0x5cb0a9dc, 0x76f988da,
        0x983e5152, 0xa831c66d, 0xb00327c8, 0xbf597fc7, 0xc6e00bf3, 0xd5a79147, 0x06ca6351, 0x14292967,
        0x27b70a85, 0x2e1b2138, 0x4d2c6dfc, 0x53380d13, 0x650a7354, 0x766a0abb, 0x81c2c92e, 0x92722c85,
        0xa2bfe8a1, 0xa81a664b, 0xc24b8b70, 0xc76c51a3, 0xd192e819, 0xd6990624, 0xf40e3585, 0x106aa070,
        0x19a4c116, 0x1e376c08, 0x2748774c, 0x34b0bcb5, 0x391c0cb3, 0x4ed8aa4a, 0x5b9cca4f, 0x682e6ff3,
        0x748f82ee, 0x78a5636f, 0x84c87814, 0x8cc70208, 0x90befffa, 0xa4506ceb, 0xbef9a3f7, 0xc67178f2}

    sha256data := [32]byte{}

    // l + 1 + m = 448 mod 512
    l := len(message) * 8
    m := (447 - l % 512) % 512
    code := make([]byte, (l + 1 + m + 64) / 8)
    copy(code[0: l / 8], message)
    code[l / 8] = byte(1 << 7)
    // code[l + 1: l + 1 + m] = byte(0)
    binary.BigEndian.PutUint64(code[(l + 1 + m + 64) / 8 - 8: (l + 1 + m + 64) / 8], uint64(l))
    N := (l + 1 + m + 64) / 8 / 64
    w := [64]uint32{}
    for n := 0; n < N; n++ {
        // compute w
        for i := 0; i < 16; i++ {
            w[i] = binary.BigEndian.Uint32(code[i * 4 + n * 64: (i + 1) * 4 + n * 64])
        }
        for i := 16; i < 64; i++ {
            s0 := rightRotate(w[i - 15], 7) ^ rightRotate(w[i - 15], 18) ^ (w[i - 15] >> 3)
            s1 := rightRotate(w[i - 2], 17) ^ rightRotate(w[i - 2], 19) ^ (w[i - 2] >> 10)
            w[i] = w[i - 16] + s0 + w[i - 7] + s1
        }

        a := h0
        b := h1
        c := h2
        d := h3
        e := h4
        f := h5
        g := h6
        h := h7
        for i := 0; i < 64; i++ {
            t1 := h + Sigma1(e) + ch(e, f, g) + k[i] + w[i]
            t2 := Sigma0(a) + maj(a, b, c)
            h = g
            g = f
            f = e
            e = d + t1
            d = c
            c = b
            b = a
            a = t1 + t2
        }
        h0 = a + h0
        h1 = b + h1
        h2 = c + h2
        h3 = d + h3
        h4 = e + h4
        h5 = f + h5
        h6 = g + h6
        h7 = h + h7
    }
    h := []uint32{h0, h1, h2, h3, h4, h5, h6, h7}
    for i := 0; i < 8; i++ {
        binary.BigEndian.PutUint32(sha256data[i * 4: (i + 1) * 4], h[i])
    }

    return sha256data
}
    \end{lstlisting}
    \item 构建\emph{Merkle树}\\
    构建Merkle树分为两部分,一部分为构建Merkle结点,一部分为构建Merkle树。
    首先实现新建Merkle结点,分为两种情况,一种是边缘的Merkle结点,即左右子孩子均为\jetbrains{nil}的结点,
    另一种为中间Merkle结点,即左右子孩子均为Merkle结点的结点。具体代码如下:\\
    \begin{lstlisting}[language=go]
func NewMerkleNode(left, right *MerkleNode, data []byte) *MerkleNode {
	node := MerkleNode{}
	if left == nil && right == nil {
		hash := mySha256(data)
		node.Data = hash[:]
	} else {
		// ... means two params
		prevHashes := append(left.Data, right.Data...)
		hash := mySha256(prevHashes)
		node.Data = hash[:]
	}
	node.Left = left
	node.Right = right

	return &node
}
    \end{lstlisting}
    对于构建Merkle树,先对每个原始数据构建一个对应的结点,然后利用这些结点进行Merkle树的构建。
    在顶层Merkle结点数大于1时,分为两步进行构建,第一步为确保该层Merkle结点数为偶数,
    第二步再利用相邻两个Merkle结点的数据构建内部Merkle结点。具体代码如下:\\
    \begin{lstlisting}[language=go]
func NewMerkleTree(data [][]byte) *MerkleTree {
    var nodes []MerkleNode
    // make merkle node for every data
    for _, dataItem := range data {
        node := NewMerkleNode(nil, nil, dataItem)
        nodes = append(nodes, *node)
    }
    // merge all nodes to a tree, merkle tree
    for len(nodes) > 1 {
        // make the number of nodes even
        if len(nodes) % 2 == 1 {
            nodes = append(nodes, nodes[len(nodes) - 1])
        }
        var tmpNodes []MerkleNode
        for i := 0; i < len(nodes); i += 2 {
            node := NewMerkleNode(&nodes[i], &nodes[i + 1], nil)
            tmpNodes = append(tmpNodes, *node)
        }
        nodes = tmpNodes
    }
    mTree := MerkleTree{&nodes[0]}

    return &mTree
}
    \end{lstlisting}
    \item 添加区块
    添加区块的本质是更新数据库和区块链的内容,即将新的区块信息添加到数据库中,并将区块链中的当前区块更新。
    具体实现如下:\\
    \begin{lstlisting}[language=go]
func (bc *Blockchain) AddBlock(data []string) {
    newBlock := NewBlock(data, bc.tip)
    bc.tip = newBlock.Hash
    err := bc.db.Update(func(tx *bolt.Tx) error {
        b := tx.Bucket([]byte(blocksBucket))
        err := b.Put(newBlock.Hash, newBlock.Serialize())
        if err != nil {
            log.Panic(err)
        }
        err = b.Put([]byte("l"), newBlock.Hash)
        if err != nil {
            log.Panic(err)
        }
        return nil
    })
    if err != nil {
        log.Panic(err)
    }
    return
}
    \end{lstlisting}
\end{enumerate}

\section{结果及分析}
\jetbrains{mySha256}函数测试结果如下:
\begin{figure}[H]
    \centering
    \includegraphics*[width=0.7\textwidth]{./figs/sha256_result.jpg}
    \caption{mySha256函数测试结果}
    \label{sha256 result}
\end{figure}
构建Merkle树部分测试结果如下:
\begin{figure}[H]
    \centering
    \includegraphics*[width=0.7\textwidth]{./figs/merkle_result.jpg}
    \caption{构建Merkle树部分测试结果}
    \label{merkle result}
\end{figure}
添加区块部分测试
\begin{figure}[H]
    \centering
    \includegraphics*[width=0.7\textwidth]{./figs/block_result.jpg}
    \caption{区块测试结果}
    \label{block result}
\end{figure}

根据图\ref{block result}的结果可以发现,数据库的数据被正常修改
(因为第一次print时就有了多的信息,说明之前退出后修改成功保存),也可以看到可以正常添加区块。\\
但是可以发现,每次添加区块,哈希值仅仅是在原来的哈希值后添加了\jetbrains{0x31},
经过查看框架代码,可以发现\jetbrains{NewBlock}函数中对哈希值的处理如下,
\begin{center}
    \jetbrains{hash := append(prevBlockHash, '1')}
\end{center}
而字符\jetbrains{1}的ascii码刚好为\jetbrains{0x31}。
由于实验没有要求对此部分进行实现,考虑到助教可能出去其他考虑特意如此实现,故没有修改。

\section{实验总结}

\subsection{收获}

\begin{itemize}
    \item 通过本次试验,我更加深入地理解了\jetbrains{SHA256}、\jetbrains{Merkle树}和区块链的基本概念,
    在自己具体通过代码实现部分功能后,不再像以前一样简单的了解。
    \item 在完成实验的同时,简单学习了\jetbrains{Go}这门语言,了解了基本语法和代码格式等。
\end{itemize}

\subsection{建议}
\begin{itemize}
    \item 可以将\jetbrains{SHA256}的实验部分单独抽出来作为一次实验,可以早一点发布,
    可以让没有学习或了解过\jetbrains{Go}语言的同学提前学习或熟悉。
\end{itemize}

\end{document}
