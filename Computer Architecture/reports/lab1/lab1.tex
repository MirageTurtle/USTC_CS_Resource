\documentclass[UTF8]{article}
\usepackage{ctex}
\usepackage{geometry}
\geometry{a4paper,scale=0.8}

\title{Verilog实验1\ 实验报告}
\author{PB19071405\ 王昊元}
\date{2022 年 03 月 26 日}

\begin{document}

\pagestyle{empty}
\maketitle
\thispagestyle{empty}  % for maketitle

\section*{实验问题回答}

\noindent
基于助教提供的设计图进行回答,且省略部分有关传入段寄存器的说明。

\begin{enumerate}
    \item 描述执行一条 XOR 指令的过程(数据通路、控制信号等)
    
    \begin{itemize}
        \item \textbf{IF}\ NPC Generator 和 IF 产生 PC,
        通过 Instruction Cache 和 IR 拿到指令( XOR rd, rs1, rs2 )。
        % \emph{Inst[19:15]} 和 \emph{Inst[19:15]} 分别是 XOR 指令的两个操作数,
        % \emph{Inst[31:25]} 是Func7,为 \emph{0000000},
        % \emph{Inst[14:12]} 是Func3,为 \emph{100},
        % \emph{Inst[6:0]} 是Opcode,为 \emph{},
    
        \item \textbf{ID}\ 从 General Register File 中读出两个操作数 Reg1 和 Reg2,
        Immediate Generate 不产生立即数。
        Inst[11:7] 为 RegDstD,传入 REG ADDR 段寄存器,写回数据时使用。
        Func7 、 Opcode 和 Func3 传入 Controller Decoder,
        控制信号 ALU Func 为异或,Op1 和 Op2 选择 Forwarding 结果(这里为寄存器读取结果),使能 RegWrite,
        WB Select 选择 RESULT 段寄存器结果(这里为 ALU 计算结果), Load NPC 选择 ALU 计算结果。
    
        \item \textbf{EX}\ OP1 Sel 和 OP2 Sel 分别选择 Reg1 和 Reg2,
        控制信号 Op1 和 Op2 ,使得从寄存器中取出的操作数经选择后作为 ALU 的操作数,
        ALU Func 为异或, ALU 执行异或操作, Load NPC 选择 ALU 计算结果,存入 RESULT 段寄存器。
    
        \item \textbf{MEM}\ WB Select 选择 RESULT 段寄存器结果(这里为 ALU 计算结果)作为 WBData。
    
        \item \textbf{WB}\ RegWrite 使能使得将 WBData 写入寄存器 WBAddr 处。
    \end{itemize}

    \item 描述执行一条 BEQ 指令的过程(数据通路、控制信号等)

    \begin{itemize}
        \item \textbf{IF}\ NPC Generator 和 IF 产生 PC,
        通过 Instruction Cache 和 IR 拿到指令( BEQ rs1, rs2, imm )。
        \item \textbf{ID}\ 从 General Register File 中读出两个待比较的数 Reg1 和 Reg2,
        生成立即数,左移一位后与 PC 相加,作为 BR Target,
        Func7 、 Opcode 和 Func3 传入 Controller Decoder,
        控制信号 Op1 和 Op2 为选择 Forwarding 结果(这里为寄存器读取结果),
        BrType 为 BEQ。
        \item \textbf{EX}\ Op1 和 Op2 选择寄存器读取结果,并与 BrType 一起传入 Branch Module,
        产生 BR 信号, BR 和 BR Target 传入 NPC Generator, Hazard Module 产生相应信号。
        \item \textbf{MEM}\ 无
        \item \textbf{WB}\ 无
    \end{itemize}
    
    \item 描述执行一条 LHU 指令的过程(数据通路、控制信号等)
    \begin{itemize}
        \item \textbf{IF}\ NPC Generator 和 IF 产生 PC,
        通过 Instruction Cache 和 IR 拿到指令( LHU rd, rs1, imm )。
        \item \textbf{ID}\ 从 General Register File 中读出待计算的数 Reg1,
        Func7 、 Opcode 和 Func3 传入 Controller Decoder,
        控制信号 Op1 为选择 Forwarding 结果(这里为寄存器读取结果), Op2 为选择立即数,ALU Func 为加法,
        ImmType 为符号扩展, Load NPC 选择 ALU 计算结果,
        使能 RegWrite, Load Type 选择内存数据, WB Select 选择扩展后的数据。
        \item \textbf{EX}\ Reg1 与扩展后的立即数相加后传入 RESULT 段寄存器。
        \item \textbf{MEM}\ result (ALU计算结果)作为 ADDR 从 Data Cache 中读取数据,
        进行扩展后作为 WBData 传入段寄存器。
        \item \textbf{WB}\ 将 WBData 写入寄存器 rd 中。
    \end{itemize}
    
    \item 如果要实现 CSR 指令(csrrw,csrrs,csrrc,csrrwi,csrrsi,csrrci),设计图中还需要增加什么部件和数据通路?给出详细说明。

    \begin{itemize}
        \item csrrw: 读后写控制状态寄存器, t = CSRs[csr]; CSRs[csr] = x[rs1]; x[rd] = t,故需要额外的寄存器来保存t。
        \item csrrwi: 立即数读后写控制状态寄存器, x[rd] = CSRs[csr]; CSRs[csr] = zimm,故需要 Immediate Operand Unit 实现5位立即数的零扩展。
        \item csrrci: 立即数读后清除控制状态寄存器, t = CSRs[csr]; CSRs[csr] = t \& \textasciitilde zimm; x[rd] = t。
        \item csrrs: 读后置位控制状态寄存器, t = CSRs[csr]; CSRs[csr] = t | x[rs1]; x[rd] = t。
        \item csrrc: 读后清除控制状态寄存器, t = CSRs[csr]; CSRs[csr] = t \& x[rs1]; x[rd] = t。
        \item csrrsi: 立即数读后置位控制状态寄存器, t = CSRs[csr]; CSRs[csr] = t | zimm; x[rd] = t。
    \end{itemize}

    如果要实现 CSR(Control Status Register) 指令,需要增加 regfile,提供给状态寄存器。
    在EX段寄存器的输入段需要加入多选器,确定读的数是来自 CSR 还是原来的寄存器堆。
    增加立即数扩展的功能及逻辑控制,因为 CSR 指令的立即数扩展与原来不同。
    在 Control Unit 中要加入 CSR 相关的控制逻辑,多选器等信号控制。
    
    \item Verilog 如何实现立即数的扩展?
    \begin{itemize}
        \item 无符号扩展:在立即数的剩余高位补零。例如:assign ex\_data = {16'h0000, imm[15:0]}。
        \item 有符号扩展:根据立即数的最高位进行扩展。例如:assign ex\_data = imm[15] ? {16'hffff, imm[15:0]} : {16'h0000, imm[15:0]}。
    \end{itemize}
    
    \item 如何实现 Data Memory 的非字对齐的 Load 和 Store?
    
    \begin{itemize}
        \item 分两次 Load 或 Store 即可。
        \item 先读取 addr \& 32'hffff\_fffc,即低位字节所在地址,然后根据 addr[1:0] 选择低位数据写入或读取;
        \item 再读取 (addr + 32'h0000\_0010) \& 32'hffff\_fffc,即高位字节所在地址,然后根据俄 addr[1:0] 选择高位数据写入或读取。
    \end{itemize}
    
    \item ALU 模块中,默认 wire 变量是有符号数还是无符号数?
    
    \begin{itemize}
        \item 无符号数。
    \end{itemize}
    
    \item 简述BranchE信号的作用。
    
    \begin{itemize}
        \item 用来指示 Branch 类型的指令是否跳转,
        Branch Module 会根据传入的 Op1、 Op2 和 BrType 来判断当前 Br 指令是否成功跳转,
        并输出相应的控制信号 BranchE。
        \item 同时 BranchE 信号还会使得 Hazard Module 产生不同的控制信号。
    \end{itemize}

    \item NPC Generator 中对于不同跳转 target 的选择有没有优先级?
    
    \begin{itemize}
        \item Br 和 Jalr 优先级最高, Jal 优先级次之, PC + 4 优先级最低。
        \item 因为 Br 和 Jalr 在 EX 阶段计算并返回信号,而 Jal 在 ID 段计算并返回信号,
        如果 NPC Generator 同时收到Br/Jalr 和 Jal 的信号( Br 和 Jalr 不会同时有效),
        则意味着 Br/Jalr 是 Jal前一条指令,则应先执行 Br/Jalr,故 Br 和 Jalr 优先级高于 Jal。
        \item PC + 4 只有在没有任何跳转发生时执行,故优先级最低。
    \end{itemize}
    
    \item Harzard 模块中,有哪几类冲突需要插入气泡,分别使流水线停顿几个周期?
    
    \begin{itemize}
        \item 数据相关冲突:如 Load 指令后执行 ALU 指令,需要插入气泡,在 ALU 计算时,操作数还未读出来,
        EX 段 stall, MEM 段 flush,故需停顿 1 个周期。
        \item 控制相关冲突:在跳转时,需要插入气泡, flush 掉 IF 段的指令,停顿 1 个周期。
        \item 条件转移:需要插入气泡, flush 掉 IF、 ID 段的指令,停顿 2 个周期。
    \end{itemize}
    
    \item Harzard 模块中采用静态分支预测器,即默认不跳转,遇到 branch 指令时,如何控制 flush 和 stall 信号?
    
    \begin{itemize}
        \item 遇到 branch 指令时,默认不跳转,当 branch 指令执行到 EX 段时, Branch Module 会产生 BranchE 信号(是否跳转)。
        如果 BranchE 有效,那么当前 ID 段的 PC + 4、 IF 段的 PC + 8 (PC 为 branch 指令的 PC)不应被执行,使能 FlushD 和 FlushE,
        清空对应段间寄存器,执行 Branch Target 指令;如果 BranchE 无效,说明跳转失败,继续执行命令即可,即不使能 Flush 信号。
    \end{itemize}
    
    \item 0 号寄存器值始终为 0,是否会对 forward 的处理产生影响?
    
    \begin{itemize}
        \item 会。
        \item 如果不判断, forward 过程中可能将 0 号寄存器的值改为非零值。
        \item 例如 XOR 0, rs1, rs2; XOR rd, 0, rs2; 此时,前一指令试图修改 0 号寄存器,但其实并不能修改,
        如果按照正常的 forward 处理,将计算结果( WBData )转发给后一条指令的第一个操作数,就会产生结果错误。
        \item 故 forward 中要加入是否为 0 号寄存器的判断,控制多选器的输出。
    \end{itemize}
    
\end{enumerate}

\section*{其他}

\noindent
如果所做回答中有任何不正确的地方,或实现不妥当有其他风险、表述有问题的地方,还烦请助教指出,
这可以让我学到更多知识,也会对我之后的实验实现有所帮助。

\end{document}