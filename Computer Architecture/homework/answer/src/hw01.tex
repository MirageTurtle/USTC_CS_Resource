\documentclass{article}
\usepackage{ctex}
\usepackage{geometry}
\geometry{a4paper,scale=0.8}
\usepackage{enumitem}
\usepackage{amsmath}
\usepackage{amssymb}

\title{HW01}
\author{PB19071405\ 王昊元}


\begin{document}
    \pagestyle{empty}

    \maketitle
    \thispagestyle{empty}  % for maketitle

    \begin{enumerate}[label=\textbf{EX\arabic*}]
        % EX1
        \item \begin{enumerate}[label=1.\arabic*]
            \item $\text{CPI} = 2 \times 30\% + 3 \times 25\% + 2 \times 20\% + 4 \times 15\% + 4 \times 5\% + 2 \times 5\% = 2.65$
            \item $T = T_{\text{CPU}} \times \text{CPI} \times \text{IC}$

            \begin{itemize}
                \item 对于A方案,CPI和IC不变,$T_{\text{CPU}}$变为原来的$0.9$倍,故$T$变为原来的$0.9$倍。
                \item 对于B方案,$T_{\text{CPU}}$和IC不变,CPI变为$2.45 (= 2 \times 30\% + 3 \times 25\% + 2 \times 20\% + 3 \times 15\% + 3 \times 5\% + 2 \times 5\%)$,
                变为原来的$\frac{49}{53}$倍,故$T$变为原来的$\frac{49}{53}$倍
            \end{itemize}
        
            $0.9 < \frac{49}{53}$,故对于该标准测试程序,A方案优化效果更好。
        \end{enumerate}
        % EX2
        \item 用$T$表示加速过后的执行时间。
        \begin{enumerate}[label=2.\arabic*]

            \item 当运算整体加速比达到3时,有

            $$\frac{(1-\alpha)T + 20 \times \alpha T}{T} = 3$$
        
            得,
        
            $$\alpha = \frac{2}{19}$$

            \item 被加速的运算在原执行时间中的比例$\beta$为
            
            $$\beta = \frac{20 \times \alpha T}{(1-\alpha)T + 20 \times \alpha T} = \frac{40}{57}$$

            \item 加速方式最大加速比在运算均被加速时达到,即20倍。运算整体加速比能达到此加速方式最大加速比的一半时有,
            
            $$\frac{(1-\alpha)T + 20 \times \alpha T}{T} = \frac{20}{2}$$

            得,

            $$\alpha = \frac{9}{19}$$
        \end{enumerate}
        % EX3
        \item \begin{enumerate}[label=3.\arabic*]
            \item 在功率相同的情况下,运行时间为最低要求速度的50\%,则可以节省50\%的能量。
            \item 频率变为原来的一半,整个计算过程时间变为原来的二倍,即运行时间与最低要求速度一样,
            但电压为原来的一半,而$\text{功率}\propto\text{电压}^2$,所以,可以节省75\%的能量。
        \end{enumerate}
    \end{enumerate}

\end{document}